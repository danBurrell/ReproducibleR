% Options for packages loaded elsewhere
\PassOptionsToPackage{unicode}{hyperref}
\PassOptionsToPackage{hyphens}{url}
%
\documentclass[
]{book}
\usepackage{lmodern}
\usepackage{amssymb,amsmath}
\usepackage{ifxetex,ifluatex}
\ifnum 0\ifxetex 1\fi\ifluatex 1\fi=0 % if pdftex
  \usepackage[T1]{fontenc}
  \usepackage[utf8]{inputenc}
  \usepackage{textcomp} % provide euro and other symbols
\else % if luatex or xetex
  \usepackage{unicode-math}
  \defaultfontfeatures{Scale=MatchLowercase}
  \defaultfontfeatures[\rmfamily]{Ligatures=TeX,Scale=1}
\fi
% Use upquote if available, for straight quotes in verbatim environments
\IfFileExists{upquote.sty}{\usepackage{upquote}}{}
\IfFileExists{microtype.sty}{% use microtype if available
  \usepackage[]{microtype}
  \UseMicrotypeSet[protrusion]{basicmath} % disable protrusion for tt fonts
}{}
\makeatletter
\@ifundefined{KOMAClassName}{% if non-KOMA class
  \IfFileExists{parskip.sty}{%
    \usepackage{parskip}
  }{% else
    \setlength{\parindent}{0pt}
    \setlength{\parskip}{6pt plus 2pt minus 1pt}}
}{% if KOMA class
  \KOMAoptions{parskip=half}}
\makeatother
\usepackage{xcolor}
\IfFileExists{xurl.sty}{\usepackage{xurl}}{} % add URL line breaks if available
\IfFileExists{bookmark.sty}{\usepackage{bookmark}}{\usepackage{hyperref}}
\hypersetup{
  pdftitle={Introductory R for Reproducible Scientific Research},
  pdfauthor={Daniel Burrell},
  hidelinks,
  pdfcreator={LaTeX via pandoc}}
\urlstyle{same} % disable monospaced font for URLs
\usepackage{longtable,booktabs}
% Correct order of tables after \paragraph or \subparagraph
\usepackage{etoolbox}
\makeatletter
\patchcmd\longtable{\par}{\if@noskipsec\mbox{}\fi\par}{}{}
\makeatother
% Allow footnotes in longtable head/foot
\IfFileExists{footnotehyper.sty}{\usepackage{footnotehyper}}{\usepackage{footnote}}
\makesavenoteenv{longtable}
\usepackage{graphicx,grffile}
\makeatletter
\def\maxwidth{\ifdim\Gin@nat@width>\linewidth\linewidth\else\Gin@nat@width\fi}
\def\maxheight{\ifdim\Gin@nat@height>\textheight\textheight\else\Gin@nat@height\fi}
\makeatother
% Scale images if necessary, so that they will not overflow the page
% margins by default, and it is still possible to overwrite the defaults
% using explicit options in \includegraphics[width, height, ...]{}
\setkeys{Gin}{width=\maxwidth,height=\maxheight,keepaspectratio}
% Set default figure placement to htbp
\makeatletter
\def\fps@figure{htbp}
\makeatother
\setlength{\emergencystretch}{3em} % prevent overfull lines
\providecommand{\tightlist}{%
  \setlength{\itemsep}{0pt}\setlength{\parskip}{0pt}}
\setcounter{secnumdepth}{5}
\usepackage{booktabs}

% Define new commands
\newcommand{\stan}{\textsc{Stan}}
<script src="https://cdnjs.cloudflare.com/ajax/libs/iframe-resizer/3.5.16/iframeResizer.min.js" type="text/javascript"></script>
\usepackage[]{natbib}
\bibliographystyle{apalike}

\title{Introductory R for Reproducible Scientific Research}
\author{Daniel Burrell}
\date{2020-12-02}

\begin{document}
\maketitle

{
\setcounter{tocdepth}{1}
\tableofcontents
}
\hypertarget{preface}{%
\chapter{Preface}\label{preface}}

Many higher degree research students and other researchers find themselves faced with the need to produce their statistical data analyses on their own. The open source R software is a popular programming language designed with statistical computing and data analysis in mind and it is commonly encouraged within academia because it is powerful, flexible and free to use and develop. It comes with access to an extensive suite of third-party packages which often implement statistical procedures and techniques that are at the cutting edge of research in various application areas. All this power and flexibility, however, comes at the cost of a fairly steep learning curve, especially when the student or researcher does not have a strong background in computer science, programming, mathematics and/or statistics. This book is aimed at assisting higher degree research students and other researchers to approach their scientific analyses in a reproducible way, emphasizing programming best practices (e.g.~breaking down analyses into modular units, task automation and encapsulation), while also teaching the use of R for basic statistical data analysis.

\hypertarget{setting-up-1.1_setup}{%
\chapter{Setting Up \{\#1.1\_setup\}}\label{setting-up-1.1_setup}}

Science is a multifaceted process that involves designing experiments or observational studies, collecting data and analyzing that data to gain insight into substantive research problems and to derive conclusions.

\hypertarget{getting-familiar-with-r-and-rstudio-1.2_getting_familiar}{%
\chapter{Getting familiar with R and RStudio \{\#1.2\_getting\_familiar\}}\label{getting-familiar-with-r-and-rstudio-1.2_getting_familiar}}

\hypertarget{getting-r}{%
\section{Getting R}\label{getting-r}}

\hypertarget{getting-rstudio}{%
\section{Getting RStudio}\label{getting-rstudio}}

\hypertarget{introduction-to-rstudio}{%
\section{Introduction to RStudio}\label{introduction-to-rstudio}}

\hypertarget{rstudio-workflow}{%
\section{RStudio Workflow}\label{rstudio-workflow}}

\hypertarget{introduction-to-r}{%
\section{Introduction to R}\label{introduction-to-r}}

\hypertarget{using-r-as-a-glorified-caclulator}{%
\subsection{Using R as a glorified caclulator}\label{using-r-as-a-glorified-caclulator}}

\hypertarget{ready-made-mathemtical-functions}{%
\subsection{Ready Made Mathemtical Functions}\label{ready-made-mathemtical-functions}}

\hypertarget{boolean-comparisons}{%
\subsection{Boolean Comparisons}\label{boolean-comparisons}}

\hypertarget{variables-and-assignment}{%
\subsection{Variables and Assignment}\label{variables-and-assignment}}

\hypertarget{vectorization}{%
\subsection{Vectorization}\label{vectorization}}

\hypertarget{managing-the-r-environment}{%
\subsection{Managing the R Environment}\label{managing-the-r-environment}}

\hypertarget{third-party-contributed-r-packages}{%
\subsection{Third-party (contributed) R Packages}\label{third-party-contributed-r-packages}}

\hypertarget{project-management-with-rstudio-1.3_project_management}{%
\chapter{Project Management with RStudio \{\#1.3\_project\_management\}}\label{project-management-with-rstudio-1.3_project_management}}

\hypertarget{getting-help-1.4_getting_help}{%
\chapter{Getting Help \{\#1.4\_getting\_help\}}\label{getting-help-1.4_getting_help}}

\hypertarget{getting-and-exploring-data-in-r}{%
\chapter{Getting and Exploring Data in R}\label{getting-and-exploring-data-in-r}}

To get data into R and to explore data in R.

\hypertarget{publication-quality-graphics-in-r-with-ggplot2-and-ggpubr}{%
\chapter{Publication Quality Graphics in R with ggplot2 and ggpubr}\label{publication-quality-graphics-in-r-with-ggplot2-and-ggpubr}}

Using ggplot2 and ggpubr to produce publication quality graphics.

\hypertarget{improving-efficiency-of-r-code-using-functions-and-vectorization}{%
\chapter{Improving Efficiency of R code using Functions and Vectorization}\label{improving-efficiency-of-r-code-using-functions-and-vectorization}}

Using functions and vectorization in R

\hypertarget{data-manipulation-with-r}{%
\chapter{Data Manipulation with R}\label{data-manipulation-with-r}}

Using dplyr to manipulate data in R

\hypertarget{producing-reproducible-reports-with-knitr}{%
\chapter{Producing Reproducible Reports with knitr}\label{producing-reproducible-reports-with-knitr}}

Reproducible reports

\hypertarget{tips-on-writing-good-r-code}{%
\chapter{Tips on writing good R code}\label{tips-on-writing-good-r-code}}

Some expert programming tips

  \bibliography{book.bib}

\end{document}
